\documentclass[a4paper,12pt, twoside]{article}
\usepackage[a4paper, left=2.5cm, right=2.5cm, top=2.5cm, bottom=2.5cm, bindingoffset=0.5cm]{geometry}
\usepackage[utf8x]{inputenc}
\usepackage{polski}
\usepackage[polish]{babel}
\usepackage{graphicx}
\graphicspath{ {C:/Users/kuban/Desktop/Inzynierka/E-Card/Praca dyplomowa/images/} }
\usepackage{indentfirst}
\usepackage{float}
\usepackage{caption}
\usepackage{adjustbox}
\usepackage{array}
\usepackage{makecell}
\usepackage{amsmath}
\usepackage{enumitem}
\usepackage{afterpage}
\usepackage{subfig}
\usepackage{hyperref}
\usepackage{url}
\renewcommand\thesection{\arabic{section}.}
\renewcommand\thesubsection{\arabic{section}.\arabic{subsection}.}
\renewcommand\thesubsubsection{\arabic{section}.\arabic{subsection}.\arabic{subsubsection}.}
\frenchspacing
\setlength{\parindent}{.5cm}
\makeatletter
\setlength{\@fptop}{0pt}
\makeatother
\linespread{1.5}
\begin{document}
	\newpage
	\thispagestyle{empty}
	\begin{center}
		
		\begin{figure}
			\centering
			\includegraphics[width=5cm]{logo_polsl.jpg}
			\vspace{.5cm}
		\end{figure}
		
		{\fontsize{17}{17}\selectfont
			\textsc{Politechnika Śląska \\[.3cm]
				Wydział Automatyki, Elektroniki i Informatyki  \\[.3cm]
				Kierunek Automatyka i Robotyka  \\[1.5cm]}
			\textbf{Projekt inżynierski \\[0.7cm]}}
		
		\Large
		{Elektroniczna plakietka konfigurowana z urządzenia mobilnego \\[4cm]}
		\Large{\begin{flushleft}
				Autor: Jakub Legutko\\
				Kierujący pracą: dr inż. Grzegorz Dziwoki\\[0.3cm]
		\end{flushleft}}
		
		\normalsize
		\vfill Gliwice, styczeń 2019
	\end{center}
	\newpage
	\newpage
	\thispagestyle{empty}
	\tableofcontents
	\newpage
	%\leavevmode\thispagestyle{empty}\newpage
	\newpage
	\clearpage
	\setcounter{page}{1}
	\section{Wstęp}
	
	\subsection{Wprowadzenie}
	Zmiany klimatyczne są coraz bardziej widoczne na świecie. Wycinka drzew jest jednym z głównych powodów wzrostu emisji gazów cieplarnianych\cite{clima_causes}, ograniczenie jej wpłynie pozytywnie na poziom ${CO_{2}}$ w atmosferze, co przełoży się na polepszenie klimatu. 
	
	Druk wizytówek oraz identyfikatorów pochłania znaczące ilości papieru, oraz tuszu w skali globalnej ze względu na ich jednorazowe wykorzystanie w większości przypadków. Z tego powodu ważne jest znalezienie alternatywnej metody. 
	
	
	\subsection{Cel projektu}
	
	
	
	\section{Podsumowanie}
	
	\newpage
	\section{Bibliografia}
	
	\begingroup
	\renewcommand{\section}[2]{}%
	\begin{thebibliography}{}
		
		\bibitem{clima_causes}
		https://ec.europa.eu/clima/change/causes\_pl\\
		Dostęp: 23 grudnia 2019
		
		
	\end{thebibliography}
	\endgroup
	
	\newpage
	\section{Spis rysunków}
	\begingroup
	\renewcommand{\section}[2]{}%
	\listoffigures
	\endgroup
	
\end{document}

